\documentclass[
    12pt, % Default font size, values between 10pt-12pt are allowed
    %spanish, % Uncomment for Spanish
]{fphw}

% Template-specific packages
\usepackage[utf8]{inputenc} % Required for inputting international characters
\usepackage[T1]{fontenc} % Output font encoding for international characters
\usepackage{fontspec,unicode-math} % Required for using utf8 characters in math mode
\usepackage{parskip}  % To add extra space between paragraphs
% \usepackage{mathpazo} % Use the Palatino font
\usepackage{graphicx} % Required for including images
\usepackage{booktabs} % Better horizontal rules in tables
\usepackage{hyperref} % For links (both internal and external)
\usepackage{listings} % Required for insertion of code
\usepackage{enumerate}% To modify the enumerate environment
\usepackage{cleveref} % Better \ref command -> \cref
\usepackage{import}   % This 4 packages and the command allow importing pdf
\usepackage{xifthen}  % figures generated with inkscape
\usepackage{pdfpages} % Source: https://castel.dev/post/lecture-notes-2/
\usepackage{mathtools}% Mathematical environments
\usepackage{relsize}  % \smaller and \mathsmaller commands
\usepackage{physics}  % A lot of good commands for mathematics
\usepackage[binary-units]{siunitx}  % To write magnitudes with units
\usepackage{multirow}
% \usepackage{multicolumn}
\usepackage{transparent}
\newcommand{\incfig}[1]{%
    \def\svgwidth{0.95\columnwidth}
    \small
        \import{./images/}{#1.pdf_tex}
}

\setlength{\parindent}{15pt}

%----------------------------------------------------------------------------------------
%    ASSIGNMENT INFORMATION
%----------------------------------------------------------------------------------------

\title{Assignment 1 \\ Optimization of a sequential application} % Assignment title

\author{Emilio Domínguez Sánchez} % Student name

\date{October 18th, 2020} % Due date

\institute{University of Murcia \\ Faculty of Informatics} % Institute or school name

\class{Arquitectura y Organización de Computadores} % Course or class name

\professor{Dr. José Manuel García Carrasco} % Professor or teacher in charge of the assignment

%----------------------------------------------------------------------------------------
%    Definitions
%----------------------------------------------------------------------------------------

\lstset{
    basicstyle=\footnotesize,        % the size of the fonts that are used for the code
    frame=L,
    captionpos=t
}
\newcommand{\tech}{\texttt}
\newcommand{\gcc}{\textit{gcc}}
\newcommand{\clang}{\textit{clang}}

\DeclareMathOperator{\performance}{Performance}
\DeclareMathOperator{\bandwidth}{Bandwidth}
\DeclareSIUnit\flop{FLOP}
\DeclareSIUnit[per-mode=fraction]\fps{\flop\per\s}
\DeclareSIUnit[per-mode=fraction]\gfps{\giga\flop\per\s}
\DeclareSIUnit[per-mode=fraction]\gBps{\giga\byte\per\s}
\DeclareSIUnit\op{operation}
\DeclareSIUnit\ins{instruction}
\DeclareSIUnit\cycle{cycle}
\DeclareSIUnit\core{core}
\DeclareSIUnit\socket{socket}
\DeclareSIUnit\node{node}
\DeclareSIUnit\query{query}
\DeclareSIUnit\channel{channel}
\DeclareSIUnit\controller{controller}

\begin{document}

\maketitle % Output the assignment title, created automatically using the information in the custom commands above

%----------------------------------------------------------------------------------------
%    ASSIGNMENT CONTENT
%----------------------------------------------------------------------------------------

\section*{Problem}

\begin{problem}

    Suppose you had a data center that was equiped with
a \textit{Haswell E5-2699 v3} general purpose processor,
an \textit{NVIDIA K80} GPU and
an TPU from \textit{Google} specifically designed for machine learning loads.
The power performance is described in \cref{tab:power}.
The performance characteristics vary depending on the type of load.
In \cref{tab:performance} you can see the details
for three different workloads $A$, $B$ and $C$.
Where $A$ and $B$ represent workloads typical of the programs that train neural networks
and $C$ represents a traditional workload.

\begin{enumerate}
    \item If the data center spends $70\%$ of its time on the workload $A$
and the rest of its time in the workload B,
what is the \textit{speedup} of using the TPUs over the GPUs? %TODO

    \item Assume that the power consumption for each processor is lineal between
its idle value and its max load value.
Specifically, we have measured that for the loads of the previous question
the general purpose processor is working at $59.4\%$ of its capacity,
the GPUs are doing it at $55.9\%$ and
the TPUs at $88.6\%$.
What is the power consumption for each of the three platforms?

    \item After some time, the data center has the following distribution of load.
$40\%$ of the load is of type $A$,
$10\%$ is of type $B$ and
$50\%$ of type $C$.
What is the \textit{speedup} of using GPUs or TPUs compared to using the platform
with general purpose processors.
\end{enumerate}

\end{problem}

\begin{table}[h]
    \footnotesize
    \centering
    \begin{tabular}{c c c c c}
        \multirow{2}{*}{System} & \multirow{2}{*}{Chip} & \multirow{2}{*}{TDP}
            & Power Consumption & Power Consumption \\
        & & & (stand-by) & (max. load) \\
        \hline \hline \\

        General purpose & \textit{Haswell E5-2699 v3} &
            $\SI{504}{\watt}$ & $\SI{159}{\watt}$ & $\SI{455}{\watt}$ \\
        GP-GPU & \textit{NVIDIA K80} &
            $\SI{1838}{\watt}$ & $\SI{357}{\watt}$ & $\SI{991}{\watt}$ \\
        ASIC & \textit{TPU(v1)} &
            $\SI{861}{\watt}$ & $\SI{290}{\watt}$ & $\SI{384}{\watt}$ \\
    \end{tabular}
    \caption{Power comsumption}
    \label{tab:power}
\end{table}

\begin{table}[h]
    \footnotesize
    \centering
    \begin{tabular}{c c c c c}
        & & \multicolumn{3}{c}{Productivity (Throughput)} \\
        System & Chip & A & B & C \\
        \hline \hline \\

        General purpose & \textit{Haswell E5-2699 v3} &
            $5482$ & $13194$ & $12000$ \\
        GP-GPU & \textit{NVIDIA K80} &
            $13461$ & $36465$ & $15000$ \\
        ASIC & \textit{TPU(v1)} &
            $225000$ & $280000$ & $2000$ \\
    \end{tabular}
    \caption{Performance parameters}
    \label{tab:performance}
\end{table}

%----------------------------------------------------------------------------------------

\newpage

\section*{Answer}

\begin{enumerate}
    \item The statement talks about the time spent in each workload,
    instead of the percentage of the workload that corresponds to a specific type.
    But then it asks about the speedup instead of the throughput increase.
    I will find the answer for the throughput thinking that the time spent is fixed
    and the speedup thinking that the percentages refer to the workload and not the time.

    \begin{enumerate}
        \item Percentages with respect to time.

        We will compute the throughput for each case:

        \begin{align*}
            \operatorname{Throughput}_{\text{GPU}} &=
            0.7 \times 13461 + 0.3 \times 36465 =
            20362.2, \\
            \operatorname{Throughput}_{\text{TPU}} &=
            0.7 \times 225000 + 0.3 \times 280000 =
            241500.
        \end{align*}

        \noindent
        The speedup of using the TPUs would be

        \begin{align*}
            \operatorname{speedup} =
            \frac{241500}{20362.2} \approx
            11.86.
        \end{align*}

        \item Percentages with respect to workload.

        The speedups for the tasks would be $\frac{225000}{13461} \approx 16.71$
        and $\frac{280000}{36465} \approx 7.68$.
        Giving a task speedup of (using \textit{Amdahl's argument})

        \begin{equation*}
            \operatorname{speedup} \approx
            \frac{1}{\frac{0.7}{16.71} + \frac{0.3}{7.68}} \approx
            12.35.
        \end{equation*}
        
    \end{enumerate}

    \item The expression for the power consumption has to be

    \begin{equation*}
        \operatorname{Power}(\text{load ratio}) =
        \operatorname{Power}_{\text{idle}} + \text{load ratio} \times
            \qty(\operatorname{Power}_{\text{max.}}-\operatorname{Power}_{\text{idle}}). \\
    \end{equation*}

    \noindent
    Hence, the power consumption for each platform can be calulated as

    \begin{align*}
        \operatorname{Power}_{\text{CPU}}(0.594) =
        159 + 0.594 \times (455 - 159) \approx
        334.82, \\
        \operatorname{Power}_{\text{CPU}}(0.559) =
        357 + 0.559 \times (991 - 357) \approx
        711.41, \\
        \operatorname{Power}_{\text{TPU}}(0.886) =
        290 + 0.886 \times (384 - 290) \approx
        373.28.
    \end{align*}

    \item The speedup with respect to using CPUs for each workload would be

    \begin{center}
        \begin{tabular}{cccc}
            System & A & B & C \\
            \hline \hline \\
            GP-GPU &
            $\frac{13461}{5482} \approx 2.46$ &
            $\frac{36465}{13194} \approx 2.76$ &
            $\frac{15000}{12000} = 1.25$ \\
            \\
            ASIC & $\frac{225000}{5482} \approx 41.0$ &
            $\frac{280000}{13194} \approx 21.2$ &
            $\frac{2000}{12000} \approx 0.1.7$
        \end{tabular}
    \end{center}

    Hence, the total speedup would be (using \textit{Amdahl's argument})

    \begin{align*}
        \operatorname{speedup}_{\text{GPU}} &\approx
        \frac{1}{\frac{0.3}{2.46} + \frac{0.1}{2.76} + \frac{0.5}{1.25}} \approx
        1.67, \\
        \operatorname{speedup}_{\text{GPU}} &\approx
        \frac{1}{\frac{0.3}{41.0} + \frac{0.1}{21.2} + \frac{0.5}{1.67}} \approx
        0.34.
    \end{align*}

\end{enumerate}

%----------------------------------------------------------------------------------------

\end{document}
